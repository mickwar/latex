\documentclass[final]{beamer}
\mode<presentation>{\usetheme{I6pd2}}
% additional settings
\setbeamerfont{itemize}{size=\normalsize}
\setbeamerfont{itemize/enumerate body}{size=\normalsize}
\setbeamerfont{itemize/enumerate subbody}{size=\normalsize}

% additional packages
%\usepackage{times}
\usepackage{amsmath,amsthm, amssymb, latexsym}
\usepackage{exscale}
\usepackage{bm}
\usepackage{graphicx}

% width fixes the caption width of every figure
%\usepackage[font=footnotesize, width=45cm]{caption}
% margin fixes the margins, so the width can still vary
\usepackage[font=footnotesize, margin=5.5cm]{caption}
\usepackage[font=footnotesize, margin=1.0cm]{subcaption}

\usepackage{booktabs, array}
\usepackage{hhline}
%\usepackage{rotating} %sideways environment
\usepackage[english]{babel}
\usepackage[latin1]{inputenc}
\usepackage[orientation=landscape,size=custom,width=200,height=120,scale=2.0]{beamerposter}
\listfiles
\graphicspath{{figs/}}
% Display a grid to help align images
\beamertemplategridbackground[1cm]

% use serif math font (like in article)
\usefonttheme[onlymath]{serif}

\newcommand{\m}[1]{\mathbf{\bm{#1}}}

\title{Gaussian Process Modeling of Modern Mass Spectrometry Computer Experimental Data}
\author{Mickey Warner, C. Shane Reese}
\institute{Brigham Young University}

\begin{document}

\begin{frame}{}
\begin{columns}[t]

%%%%% Left Column
\begin{column}{.28\linewidth}
    %%%%% Abstract
    \begin{block}{Abstract}
    In physical chemistry, variable electron and neutral density attachment mass spectrometry (VENDAMS)
    is a new approach to measuring chemical reaction rate constants. Such reactions may be very complex,
    consisting of a set of many differential equations; computer models have been developed to simultaneously
    solve the equations. We use Gaussian process modeling to combine the field experimental and computer
    experimental output. As the VENDAMS data are measures of relative abundance, we use a logit
    transformation to preserve the normalized count constraints.
    \end{block}

    %%%%% Intro - Description of the data
    \begin{block}{Introduction}

    \begin{sc}Mass Spectrometry\end{sc}

    $\bullet$ Sample of charged particles sent through meter long tube

    $\bullet$ Particles undergo chemical reactions, separation occurs

    $\bullet$ Mass spectrometer measures the amount of each separated anion

    \bigskip
    \bigskip
    \bigskip
    \bigskip
    \begin{sc}Field Trials\end{sc}

    $\bullet$ Measurements are made on relative abundances of anions

    $\bullet$ Bounded between 0 and 1

    $\bullet$ Responses sum to 1

    $\bullet$ Recorded at several initial electron concentrations
    
    \bigskip
    \bigskip
    \bigskip
    \bigskip
    \begin{figure}
        \includegraphics[scale=1]{dataplot2.pdf}
        \caption*{Anions $Br^-$ (blue), $F^-$ (red), and $CF_3^-$ (green). Black dots (connected by thick solid lines) indicate field measurements. Thin lines represent output from the computer simulator.}
    \end{figure}
    \bigskip
    \bigskip
    \bigskip
    \bigskip

    \begin{sc}Computer Simulations\end{sc}

    $\bullet$ Solves for relative abundances given ``known'' rate constants

    $\bullet$ Inexpensive to perform simulations

    $\bullet$ Deterministic (always returns the same answer for the same input)

    \end{block}
\end{column}

%%%%% Middle Column
\begin{column}{.40\linewidth}
    %%%%% Gaussian Processes
    \begin{block}{Gaussian Processes}
    A Gaussian process (GP) is collection of random variables, any finite number of which have joint Gaussian distributions. They require a mean function $m(\m{x})$ and covariance function $k(\m{x},\m{x}')$

    \bigskip
    \bigskip
    \bigskip
    \bigskip
    \begin{figure}
        \begin{subfigure}{0.40\textwidth}
            \centering
            \includegraphics[scale=1.1]{gp_prior.pdf}
%           \caption*{Random draws from a prior Gaussian process.}
        \end{subfigure}%
        \begin{subfigure}{0.40\textwidth}
            \centering
            \includegraphics[scale=1.1]{gp_data.pdf}
%           \caption*{A Gaussian process conditioned on observed data (noise free).}
        \end{subfigure}
%       \caption*{This Gaussian process has mean function $m(x_i) = 0.5$ and covariance function $k(x_i,x_j)=\exp(-(x_i-x_j)^2/2l^2)$, for all $i$ and $j$.}
    \end{figure}
    \bigskip
    \bigskip
    \bigskip
    \bigskip

    Gaussian processes are:

    $\bullet$ Highly flexible in modeling random functions

    $\bullet$ Suited well for high dimensional designs

    $\bullet$ Great at interpolation of untested design points

    \end{block}


    %%%%% Statistical Model
    \begin{block}{Statistical Model}

    The relative abundances of observation $i$, at electron concentration $\m{x}_i$ and calibrated simulator inputs $\m{\theta}$, are modeled by
    \smallskip
    \[ y_i = \eta(\m{x}_i, \m{\theta}) + \delta(\m{x}_i) + \epsilon(\m{x}_i) \]
    \bigskip

    where $\eta(\cdot)$ is the computer simulator output, $\delta(\cdot)$ is the difference between field trails and the simulations, and $\epsilon(\cdot)$ is an error term. Independent zero-mean Gaussian processes are used to model $\eta$ and $\delta$. We use the product covariance function having power exponential form
    \bigskip
    \[ \mathrm{Cov}((\m{x},\m{t}),(\m{x}',\m{t}')) = \frac{1}{\lambda} \prod_{k=1}^{p_x} \rho_{\eta k} ^{4(x_k-x_k')^2}\times \prod_{k=1}^{p_t}(\rho_{\eta,p_x+k})^{4(t_k-t_k')^2} \]
    \bigskip

    The pair $(\m{x}, \m{t})$ are scaled to the unit hypercube $[0,1]^{p_x+p_t}$; $\m{t}$ represents simulator inputs (chemical reaction rates), used to estimate the calibrated inputs $\m{\theta}$.
    \bigskip
    \bigskip
    \bigskip
    \bigskip

    The parameter $\lambda$ is spatial precision and $\m{\rho}$ controls the dependence strength of the component directions.
    \bigskip
    \bigskip
    \bigskip
    \bigskip

    Model parameters are estimated using the Metropolis-Hastings algorithm.

    \end{block}
\end{column}

%%%%% Right Column
\begin{column}{.28\linewidth}
    %%%%% Results -- predictions
    \begin{block}{Prediction Results}

    \begin{figure}
        \centering
        \includegraphics[scale=1.11]{combine.pdf}
        \caption*{Left: predictions from the simulator. Center: discrepancy between reality and the simulator. Right: calibrated model predictions (includes simulator and discrepancy term). Solid lines are means, dotted lines are 95\% interval bounds, and solid black dots are field trials.}
    \end{figure}
    \bigskip
    \bigskip
    \bigskip
    \bigskip

    $\bullet$ Simulator does well at emulating the physical process

    $\bullet$ Most contributory reactions are accounted for in the simulator

    $\bullet$ The statistical model is validated
    
    \end{block}

    %%%%% Results -- inputs
    \begin{block}{Chemical Reaction Results}

    \begin{figure}
        \begin{subfigure}{0.44\textwidth}
            \centering
            \includegraphics[scale=1.25]{boxplot.pdf}
            \caption*{Input sensitivity for both principal components}
        \end{subfigure}%
        \begin{subfigure}{0.49\textwidth}
            \centering
            \includegraphics[scale=1.15]{matrix.pdf}
            \caption*{Univariate and bivariate plots for the first five inputs}
        \end{subfigure}
    \end{figure}
    \bigskip
    \bigskip
    \bigskip
    \bigskip
    \bigskip
    \bigskip
    \bigskip
    \smallskip

    $\bullet$ There are 9 chemical reaction rates estimated in this example

    $\bullet$ In the box plot, values away from 0 and 1 signify sensitive simulator inputs to the output

    $\bullet$ Inputs 5, 8, and 9 contribute most to the model

    $\bullet$ Inputs 1, 4, and 6 have slight effects

    $\bullet$ The histograms show the marginal posterior distributions for $\theta_1,\cdots,\theta_5$

    $\bullet$ The off-diagonals in the right figure are of joint posterior distributions.

    \end{block}
\end{column}

\end{columns}
\end{frame}
\end{document}
